\chapter*{主题概要}

\epigraph{\bfseries No scales. Never. Exercises? Never. \dots \par I was training my works and if it doesn't sound I kept studying it again. \par Technique skills doesn't fly away.}{$-$\it{Martha Argerich}}

数学分析是一门由多个不同部分组成的、扎根于十七世纪前的自然科学与自然哲学研究中的学科,其大多结果经过几个世纪的打磨,已经能非常成熟地进行数学公理化了。现代数学分析在某种程度上是对“\it{古典分析学}”进行的拓展、特化、抽象,但其本质仍然是继承着旧时代研究方法与先贤思辨的、精致而庞大的系统。

按部就班地按照有多年教学经验的教授所撰写书籍来学习,远比一头扎入名为“现代数学”的深谭要能更快去到前沿。

数学分析的教科书少说也有小30本,笔者对其中的一些有所了解,但更多的还尚未读过,便借着写笔记的机会把某些经典教材好好读读。在宏观角度下,数学分析便是严格化的\it{微积分},更确切地说是为一种\bf{微分-积分}过程提供语言描述。

在研究运动学的时候,牛顿发现将一个较为复杂的运动给分解为\bf{无穷级数}后求解非常容易计算,约在1665年时利用\it{二项式定理}得到了非常经典的结果:\it{曲线$y=ax^\frac{n}{m}$围出的面积是$\frac{an}{m+n}x^\frac{m+n}{n}$},即对函数$f(x)=ax^\frac{n}{m}$求积分。但是他所用的方法并没有很好的严谨与简洁性\footnote{具体参考《\it{微积分的历程:从牛顿到勒贝格}》的24-28页},不过对于那个仅仅能对非常简单的运动进行分析的时代来说,已经是一项非常特出的贡献了。

海峡另一侧,德国通识家莱布尼茨在1684年的时候发布了他在微积分学上工作的第一篇论文,题为《一种求极大极小值与切线的方法,适用于有理与无理量,与该特殊类型的计算》,内容关于要如何用微分求出一个“变换”的极值。在此之后,他为当时的所有微积分创建了新记号与名字,如称\it{变换}为\it{函数}、积分号$\int$、微分号$\frac{\ddy}{\ddx}$及其它。他的著作《微分学的历史与起源》有非常突出的历史地位,在文学上的造诣让这学说流传甚广,虽然历史并没有为他开拓通往未来的航道,如果你有兴趣,请去寻找名叫\bf{莱布尼茨全集}的系列。

