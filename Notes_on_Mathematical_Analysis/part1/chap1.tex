\chapter{分析浅引:空间的构筑}

\section{基本记号}

学习数学的时候,能提出问题才有结果。设想自己没有对任何事物的感受能力,仅仅能知道\it{有}“\it{这么一些东西}”存在于那里,这时能做到的事不过是将这些东西视为一个\it{收集}中的\it{不同}事物而已。

\begin{definition}
	
	一堆\bf{对象}的\bf{收集}$X$,是这一堆东西的整体。
	
\end{definition}

\begin{remark}
	
	为了定义“收集”,需要引入“集合”:\it{\bf{集合}是具有某种特定性质的对象汇集而成的一个整体, 这些对象称为该集合的\bf{元素}},但是这与收集的概念在本质上是几乎一样的。形式上,我们对收集中的对象蒙蔽了感官,所以并不能判断其是否有相同\bf{性质},在集论中有配套的\it{逻辑系},用这个系统进行性质的具体刻画是必要的。
	
\end{remark}

逻辑系统的主要目的是判断一个问题是否\bf{可解},或称\it{该命题是否为真}。从一个已知命题推出新的命题是\bf{逻辑演绎}的唯一方法,称推导第一个步骤的假设为\bf{公理},它们是\bf{恒真式}的例子,在任何附加条件下都是正确的。

\begin{definition}
	
	一些基础逻辑记号规范:
	
	\begin{enumerate}
		
		\item \tbf{真值}是定下的两个对象$\top,\bot$或$\zero,\one$,它们在所论收集$\Gamma$内作为\tbf{真}与\tbf{假},是初始的\rm{恒真式}与\rm{矛盾式}。
		
		\item 符号$p,q,r$\it{暂时}作为收集$\Gamma$中的对象,称为\tbf{命题变量},简称\tbf{变量}。
		
		\item 如果$q$是$p$的\tbf{推论},则记为$p\Rar q$,当$p$为真时$q$必然为真,其它情况则不一定。另外将恒真式记为$\vDash p$,即$p\Rar \one$。
		
		\item 对推论进行判断,记为$p\rar q$,如果$p$为真时$q$为假,则$(p\rar q)$为假,其它情况为真。
		
		\item 如果$p$与$q$互为推论,亦即$p\Rar q$且$q\Rar p$,那么它们的真值一定相同,记为$p\Lrar q$,称它们\tbf{等价}。同时简单地把真值相同的变量记为$p=q$。
		
		\item 将$p\rar \zero$记为$\neg p$,称为$p$的\tbf{否定},它只在$p$为$\zero$时为真,即它反转$p$的真值,立见$\neg \neg p = p$。
		
	\end{enumerate}
	
\end{definition}

想要操作逻辑符号,那必须进行两个逻辑词\it{之间}的关系判断,比如“当$p$与$q$都为真时该命题为真”或“当$p$与$q$中任一者为真时该命题为真”。将“该命题”专门提出来,称为\bf{与}、\bf{或},它们与上述的$\rar, \zero, \one, \neg$放在一起被称为\bf{逻辑符号}。

\begin{definition}[De Morgan律与合/析取]
	
	\tbf{合取}记为$p\wedge q = \neg(p\rar \neg q)$,\tbf{析取}则是$p\vee q = \neg p\rar q$,计算得$\neg(p\wedge q) = \neg p \vee \neg q$。
	
\end{definition}

能直接看出,若$p$非真,则其为假,称此现象为\bf{排中律},与$\neg(\neg\neg p) = \neg p$等价。\bf{反证法}是根据排中律设计的一种方法:$p$与$\neg p$中总有一个为假,想证明$p$为真;如果$\neg p\Rar r$,但$r\Rar p$,这导致\bf{悖论}$\neg p\Rar p$,根据基本假设这是谬误的,因此$p$为真。

接下来将直接使用平常人都知道的集合论语言进行描述,所谓的“收集”仅是不声明所选择的“集合论”与“逻辑系”时选择的词汇,在\it{范畴论}中会经常遇到。

\section{自然数}

可以尝试把一个集合中的元素排成序,$P = (P,\leqslant)$,比如集合$\{x,y\}$有两个不同的排序$x\leqslant y$与$y\leqslant x$,而$\{x,y,z\}$有四个,$\{w,x,y,z\}$有八个。总的来看,有$n$个元素的集合将有$2^n$个不同的排序方式,当$n\rar\infty$时,这些方式将会远远多于集合的元素。但是对于一个有限的$n$,总能选择一个使得给定元素$x$比任何其它元素更小的序,称它为\tbf{良序}。

但是没有人喜欢选择。

如果有一种\it{典型}的方式,在一个集合上给出每个元素都可排序且总有最小的元素的序,那么世界会变得简单,\bf{公理化集合论}的目的在本质上就是这样,\it{避免一切悖论,囊括全部情形}。不过直接给出公理是很难受的,就像学习向量空间的时候说“\it{向量空间是一个使得其元素可加减、可倍率且有零的集合}”一样,非常不知所云。

\begin{definition}[定理-命题-推论]
	
	所谓\tbf{定理},记为$\Gamma\vdash p$,是一个有限定范围的恒真式,范围是$\Gamma$。而\tbf{命题}是一个待证明为真的\tbf{公式},后者指由逻辑符号连接变量构成的整体,如$p\wedge \neg q$就是一个公式。\tbf{合式公式}是有意义的公式,咱$p\rar \neg p$就没有意义。能证明的公式就是合式公式。而\tbf{推论}无非是$p\Rar q$。
	
\end{definition}

一般来说省略显然命题的证明,直接称其为定理。虽然在一些意义上它并不显然。

\begin{theorem}[空集]
	
	对于任意一个给定的集合$X$,有唯一的\tbf{空集}为$\emptyset := \{x\in X\setvert x\not\in x\}$,并且$\emptyset^\cal{C} = X$。
	
\end{theorem}

\begin{definition}[幂集]
	
	给定集合$X$,可以生成新的集$P(X) := \{Y\subset X\}$,那么可得$P(\emptyset) = \{\emptyset\}$,$P(P(\emptyset)) = \{\emptyset, \{\emptyset\}\}$。
	
\end{definition}

\begin{definition}[Cartesian积]
	
	定义\tbf{有序对}$(x,y) := \{x,\{x,y\}\}$,集合$X\times Y := \{(x,y)\setvert x\in X, y\in Y\}$被称为$X$与$Y$的\tbf{Cartesian积}。
	
\end{definition}

$X\vee Y=\emptyset$则乘积为空,而且$X\times Y\subset P(P(X\cup Y))$,$Y\times X$同理。比如在$X=\{x\},Y=\{y\}$时,$ P(P(X\cup Y)) \supset P(\{x,y\}) \supset \{x,y,\{x,y\}\} $,立见其无序性。

\begin{theorem}[序数的后继]
	
	如果$\alpha\subset P(\alpha)$,那么$P(\alpha) = \alpha \cup \{\alpha\}$,称其为\tbf{序数},而序数的幂集为其\tbf{后继}。
	
\end{theorem}

可见$\emptyset$是最小的序数,所有的序数都被它生成。记$\alpha$的后继为$\alpha^+$。

\begin{proposition}[归纳法]
	
	如果命题$p$基于序数$\alpha$,记为$p=p(\alpha)$,同时$\vDash p(\emptyset)$,且对于任意序数$\beta \subsetneq \alpha$有$\vdash p(\beta)$,那么$\vdash p(\alpha)$。
	
	\begin{proof}[采纳 J. Schl\"{o}der - Ordinal Arithmetic 中的证明]
		
		假设$\neg p(\theta)$是使得该命题为假的最小序数,则$\theta \neq \emptyset$;若$\theta = \gamma^+$,由于$\vdash p(\gamma)$,有$\vdash p(\theta)$,得到悖论。
		
	\end{proof}
	
\end{proposition}

\begin{theorem}[Peano公理]
	
	可以看出$\On := $\tbf{全体序数的收集}满足下列性质:
	
	\begin{itemize}
		
		\item $ \emptyset \in \On $.
		
		\item $ \alpha \in \On \Rar \alpha^+ \in \On $.
		
		\item $ ( \alpha = \beta ) \Lrar ( \alpha^+ = \beta^+ ) $.
		
		\item $ \forall \alpha \in \On, \alpha^+ \neq \emptyset $.
		
		\item $ \vdash p(\emptyset) \wedge ( \forall \beta < \alpha, \; \vdash p(\beta) ) \Rar \; \vdash p(\alpha) $.
		
	\end{itemize}

\end{theorem}

从空集生成的序数类记为$\NN$,称之为\bf{自然数}。它包含了全部$P(\emptyset),P(P(\emptyset)),P(P(P(\emptyset))),\ldots$,直接记为$1,2,3,\ldots$,但它本身是否是集合不能从上述运算中导出,为了保持数学上的严谨性,我们规定\bf{自然数的收集是个集合}(\it{无穷公理}的一种表达)。

序数类不会是集合,因为每个序数都有后继,自然数也有后继,假设序数类包含了全部序数那它本身也是一个序数。这很显然能导出悖论:\bf{Russell悖论}的一种形式,另一种比较熟知的是\bf{全体集合不构成集合},因为$\On$在全体集合的收集中,所以它不可能是集合。

更多公理集合论、自然数构造与无穷序数的信息可以查询附录。


\section{数}

\begin{center}
	
	\it{实数是一个对加减乘除封闭、任何点之间都没有空隙的空间。}
	
\end{center}

%我们这样定义实数:

\begin{definition}
	
	实数集$\RR$满足以下条件:
	
	\begin{itemize}
		
		\item 有操作$+:(x,y)\rar x+y\in\RR$,$\times:(x,y)\rar x\cdot y\in\RR$,称为\tbf{加法}与\tbf{乘法}。
		
		\item 有序关系$(\RR,\leqslant)$,使其任意元素皆可排序,且任意非空子集总有最小元。
		
	\end{itemize}
	
	称$(\RR,+,\times,\leqslant)$为\tbf{实数系},它满足下述公理:$x,y,z$是$\RR$中的任意元素,
	
	\begin{enumerate}[F.1]
		
		\item $x+(y+z)=(x+y)+z,\; x+y=y+x,\; 0+x=x+0=x,\; x+(-x)=0$.
		
		\item $x\cdot(y\cdot z)=(x\cdot y)\cdot z,\; x\cdot y=y\cdot x,\; 1\cdot x=x\cdot 1=x,\; x\cdot (x^{-1})=1$.
		
		\item $x\cdot(y+z)=x\cdot y+x\cdot z$.
		
	\end{enumerate}
	
	\begin{enumerate}[O.1]
		
		\item $x\leqslant y\leqslant z \Rar x\leqslant z$.
		
		\item $x\leqslant y\leqslant x \Rar x = y$.
		
		\item $x\leqslant y \Rar x+z\leqslant y+z$.
		
		\item $0\leqslant x,y \Rar 0\leqslant xy$.
		
		\item $0<x \Rar \exists n,\, y<n\cdot x$.
		
		\item $P\subset \RR \Rar \exists! \sup(P)\in\RR$.
		
	\end{enumerate}
	
\end{definition}

\begin{definition}[确界,开闭性]
	
	称$x$为有序集$P\subset P'$的\tbf{上界},若$P\leqslant x$,即$\forall y\in P,\, y\leqslant x$。而\tbf{上确界}$\sup(P)$是全体上界中最小的那个,如果$P=P'$,则上确界就是$P$的\tbf{极大元}:没有比它更大的数。
	
	反过来有下确界。%如果$x,y$都是$P$的上/下确界,那它们一定相等。
	称不包含上与下确界的集合为\tbf{开集},反之是\tbf{闭集}。
	
\end{definition}

\begin{remark}
	
	指出几点重要性质:以上定义的各种概念都是唯一的,比如
	
	\begin{enumerate}
		
		\item $x+x'=x+y\Rar (-x)+x+x'=(-x)+x+y \Rar x'=y$,乘法情形相似;所以单位元都是唯一的。
		
		\item $0\leqslant x \Rar -x\leqslant x+(-x)=0$.
		
		\item $0\cdot x = (0+0)\cdot x = 0\cdot x + 0\cdot x = 0$.
		
		\item 上确界显然唯一。
		
		\item $0\leqslant x\leqslant 1\Rar 0\leqslant x\cdot x^{-1} = 1 \leqslant x^{-1}$.
		
	\end{enumerate}
	
\end{remark}

现在能够操作的数是$0,1$,它们与上述条件将能够非常有效地造出整个实数系:

\begin{definition}[自然数]
	
	只考虑加法与乘法交换结合性,可得$1^+ = 1+1 \in \NN$,$n^+ = n+1 \in \NN$。$\NN$也对数乘封闭,因为$n\times m=m+m+\ldots+m\in\NN$,写$(\NN,+,\times,1,0)$为自然数系。
	
\end{definition}

\begin{definition}[整数]
	
	在自然数的基础上考虑加法逆元,存在$1+(-1)=0,\, -1\in\ZZ$,对数乘的封闭性直接得到$-n=(-1)n\in\ZZ$,类似地,写整数为$(\ZZ,+,\times,-1,1,0)$。
	
\end{definition}

\begin{definition}[有理数]
	
	在整数中添入乘法逆元,由乘法封闭性得到$m\cdot n^{-1}\in\QQ$,记为$(\QQ,+,-,\cdot,/)$。
	
\end{definition}

\begin{proposition}[有理数的非完备性]
	
	定义$\QQ$中的\tbf{开区间}为$(a,b):=\{x\in\QQ\setvert a<x<b\}$,而\tbf{闭区间}显然是$[a,b]:=\{x\in\QQ\setvert a\leqslant x\leqslant b\}$。
	
	考虑半开半闭区间$\{[-a,x)\subset\QQ\setvert x^2<2\}$,它在\QQ 中没有上确界。
	
	\begin{proof}[陈天权 《数学分析讲义》 第一册 pp.43]
		
		取上确界$p/q\geqslant 1$,欲知$(p/q)^2\neq 2$,即$p^2=2q^2$,因此$p$是偶数。假设$p,q$互素。取$p=2r$,得到$2r=q^2$,即$q$也是偶数,那么$p/q$可以分解,与假设矛盾。
		
	\end{proof}
	
\end{proposition}

有理数与实数只差了一条公理:\tbf{完备性公理},即所谓\trm{确界原理}。

\begin{definition}[实数]
	
	根据确界原理,我们可以形式地定义实数:$\RR = \{\sup(U)\setvert U\subset \QQ\}$。
	
\end{definition}

所以实数是一个没有洞的空间,因为所有的洞都(形式地)被填上了。但是这样的公理对于研究来说只有证明上的作用,实际的构造性结果还需要更丰富的结构。

\begin{definition}[Dedekind分割]
	
	令$X\subset\QQ$,$X'=\QQ\setminus X$,并且
	
	\begin{enumerate}
		
		\item $X,X'\neq\emptyset$.
		
		\item $\forall x\in X,\, \forall x'\in X',\, x<x'$,简写为$X<X'$.
		
		\item $\nexists\max(X)$.
		
	\end{enumerate}
	
	则称$X\cup X'$为一个\tbf{分割}。
	
\end{definition}

\begin{example}
	
	考虑对应于$\sqrt{2}$的分割:$X=\{x\in\QQ_{\geqslant 0}\setvert x^2<2\}\cup \QQ_{\leqslant 0}$,那么它等价于$X=(-\infty,\sqrt{2})$,$X'=(\sqrt{2},\infty)$,显然,它定义了点$\sqrt{2}$。
	
\end{example}

取集合上的序关系$X\leqslant Y \Lrar X\subset Y$,据定义,它是一个良序(请写出证明)。我们希望能直接从分割的定义导出确界原理。

\begin{proposition}[确界原理]
	
	取$\cal{R}$为全体分割的集合,$\cal{X}\subset\cal{R}$,并且有$M\in\cal{R}$使得$\forall X\in\cal{X},\, X\leqslant M$,那么
	
	$$ \sup(\cal{X})=\bigcup_{X\in\cal{X}}X. $$
	
	\begin{proof}[于品 - 《数学分析讲义》 pp.31]
		
		第一步:证明$M_0 := \sup(\cal{X})$是个分割。
		
		\begin{enumerate}
			
			\item $M_0 \neq \emptyset \wedge M_0' \neq \emptyset$,因为$\emptyset \notin \cal{R}$。
			
			\item $\forall x_1\in M_0,\, x_2>x_1\Rar x_2\in M_0$:条件使得$\exists X\in \cal{X},\, x_1\in X$,即$x_1<x_2\Rar x_2\in X\subset M_0$。
			
			\item $\nexists\max(M_0)$。假设$x_0=\max(M_0)$,则存在$X\in\cal{X}$,$x_0=\max(X)$,与定义矛盾。
			
		\end{enumerate}
		
		第二步:说明$\sup(\cal{X})$的确是其上确界。
		
		定义$\cal{M}:=\{M\in\cal{R}\setvert \forall X\in\cal{X},\, X\leqslant M\}$为$\cal{X}$的上界族,欲证$M_0\in\cal{M}\wedge\forall M\in\cal{M},\, M_0\leqslant M$。
		
		第一步已经说明$M_0\in\cal{M}$。而作为上界,$\forall X\in\cal{X},\, X\subset M$,故$\bigcup X=M_0\subset M$,得证。
		
	\end{proof}
	
\end{proposition}

\begin{proposition}
	
	为了说明这样构造的确界原理确实与前述的完备性公理等价,我们需要证明$\cal{R}=\RR$。
	
	\begin{proof}
		
		需要证明的无非是$\cal{R}$上有加减乘除,有序,有Archimedes性质,且诸运算互相匹配。
		
		简单说明加法:$X+Y := \{x+y\in\QQ\setvert x\in X,\, y\in Y\}$,逐点计算说明加法交换结合有逆,称$(\cal{R},+)$为一个\tbf{交换群},或称\tbf{Abel群}。乘法有相似定义,逐点计算表明$(\cal{R},\cdot)$是个交换群,同时它们分配。
		
		例:$\overline{0} := \{z\in\QQ\setvert \forall z<0\}$,$\overline{0}+X = \{z+x\in\QQ\setvert \forall z<0,\, x\in X\}$,需要说明$\overline{0}+X=X$。
		\begin{enumerate}
			\item 由于$z+x<x$,$\overline{0}+X\subset X$。
			
			\item 由于$x\in X\Rar \exists \overline{x}\in X,\, x<\overline{x}$,令$z=\overline{x}-x$,则$0<z \Rar x=x'-z\in\overline{0}+X$,所以$X\subset\overline{0}+X$。
		\end{enumerate}
		
		说明序与加法互相匹配在技术上是挺复杂的,我们引入下述引理:
		
		\begin{lemma}\label{lemma:1.1}
			
			对于分割$X\cup X'$,$\forall n\in\NN$,$\exists x'\in X',\, \exists x\in X$ 使得 $\displaystyle0<x'-x<\frac{1}{n}$.
			
			\begin{proof}[于品 - 《数学分析讲义》 pp.33]
				
				取$x_0\in X,\, x_0'\in X'$,由归纳法,假设存在$x_k,x_k'$,令$y=\frac{1}{2}(x_k+x_k')$,如果$y\in X$则令$(x_{k+1},x_{k+1}')=(y,x_k')$,反之则$(x_{k+1},x_{k+1}')=(x_k,y)$,其中$k\geqslant 0$。那么
				
				$$ x_k'-x_k = \frac{1}{2}(x_{k-1}'-x_{k-1}) = \ldots = \frac{1}{2^k}(x_0'-x_0). $$
				
				显然$x_k'-x_k<x_{k-1}'-x_{k-1}$,由Archimedes性质,任选$n$有$\displaystyle\frac{1}{n}>\frac{1}{2^{k_0}}(x_0'-x_0)$。得证。
				
			\end{proof}
			
		\end{lemma}
		
		首先,$X<Y\Rar X+Z\leqslant Y+Z$,欲知$X+Z\neq X+Z$。取$y,\overline{y}\in Y-X$如上,使$\frac{1}{n}<y-\overline{y}$,另取$\overline{z}-z<\frac{1}{n}$。如果$y+z\in X+Z$,则有$y+z=x+z'$,但$x<y$为定义,那么$z'<z$,且$x<\overline{y}$与$z'<\overline{z}$。得到
		
		\begin{align*}
			x<(\overline{y}-y)+y, & \\
			z'<(\overline{z}-z)+z & < \frac{1}{n}+z < (\overline{y}-y)+z
		\end{align*}
		
		相加,得到$x+z'<y+z$,与上式矛盾。
		
		余下的都可以按照相似手段得到结果,请\tbf{读者自证}。
		
	\end{proof}
	
\end{proposition}

所以有理数上的戴德金分割与实数是等价的,还有几种从不同角度切入的实数定义,它们各有特色,在数学分析中其作用也时常有所体现。在初步课程中按照一种循序渐进的方式逐步介绍各种实数定义,因为最基本的例子都需要使用它们。


\section{点列极限}

例如在引理\ref{lemma:1.1}中,用于证明的\it{极限思想}说明每一个数都是逼近过程的“最终”结果。在一个数系中,用\bf{列}逼近希望被导出的数是非常重要的技术手段,它与戴德金分割的思想有许多相似之处,之后会见到的闭区间套、有限覆盖、Stone-Weierstrass逼近定理之流与极限思想在某些层面上都是等效的。


\section{极限的计算}




\section{线性空间}




\section{函数极限}




\section{度量空间与平移}




\section{线性映射与算子}




\section{求和、连续、收敛性判别}






